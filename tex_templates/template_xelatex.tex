% general and Chinese configuration
\documentclass[12pt]{article}
\usepackage{xeCJK}% use Latin font whenever possible
\usepackage{fontspec}% set Chinese fonts, as follows
\setCJKmainfont[BoldFont=PingFang SC,ItalicFont=STKaiti]{STSong}
\setCJKsansfont[BoldFont=STHeiti]{STXihei}
\setCJKmonofont{STFangsong}
\XeTeXlinebreaklocale "zh"% for word wrap of Chinese
\XeTeXlinebreakskip = 0pt plus 1pt% ibid., cont.
%
\usepackage[
  top=1.1cm, bottom=0.9cm, left=1.8cm, right=1.8cm
]{geometry}% sets page margins
\usepackage{longtable}% so that a table breaks across pages
\renewcommand{\arraystretch}{1.4}% increase row height of table for Chinese, often too crowded
% abbreviations
\newcommand{\It}{\textit}% simply abbrev.
\newcommand{\Bf}{\textbf}% simply abbrev.
\newcommand{\Tt}{\texttt}% simply abbrev.
\newcommand{\Sf}{\textsf}% simply abbrev.
\newcommand\Id\indent% "I"n"d"ent
% custom section, subsection
\newcommand{\mySec}[1]{%
  \setcounter{cntSubSec}{0}% comment out if counting continues every section
  \section{#1}%
}
\newcommand{\mySubSec}[1]{%
  \addtocounter{cntSubSec}{1}%
  \subsection*{\arabic{cntSubSec}.#1}%
}
\usepackage{tikz}% for producing vector graphics
\newcommand*{\Cite}[0]{
  \addtocounter{cntNote}{1}%
  \tikz[baseline=(char.base)]{
    \node[shape=circle, draw, inner sep=0.7 pt] (char) {\arabic{cntNote}};
  }
}

